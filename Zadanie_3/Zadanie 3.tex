% !TEX TS-program = pdflatex
% !TEX encoding = UTF-8 Unicode

% This is a simple template for a LaTeX document using the "article" class.
% See "book", "report", "letter" for other types of document.

\documentclass[11pt]{article} % use larger type; default would be 10pt

\usepackage[utf8]{inputenc} % set input encoding (not needed with XeLaTeX)
\usepackage[QX]{fontenc}
\usepackage{lmodern}

%%% Examples of Article customizations
% These packages are optional, depending whether you want the features they provide.
% See the LaTeX Companion or other references for full information.

%%% PAGE DIMENSIONS
\usepackage{geometry} % to change the page dimensions
\geometry{a4paper} % or letterpaper (US) or a5paper or....
% \geometry{margin=2in} % for example, change the margins to 2 inches all round
% \geometry{landscape} % set up the page for landscape
%   read geometry.pdf for detailed page layout information

\usepackage{graphicx} % support the \includegraphics command and options
\usepackage{listings}
% \usepackage[parfill]{parskip} % Activate to begin paragraphs with an empty line rather than an indent

%%% PACKAGES
\usepackage{booktabs} % for much better looking tables
\usepackage{array} % for better arrays (eg matrices) in maths
\usepackage{paralist} % very flexible & customisable lists (eg. enumerate/itemize, etc.)
\usepackage{verbatim} % adds environment for commenting out blocks of text & for better verbatim
\usepackage{subfig} % make it possible to include more than one captioned figure/table in a single float
% These packages are all incorporated in the memoir class to one degree or another...

%%% HEADERS & FOOTERS
\usepackage{fancyhdr} % This should be set AFTER setting up the page geometry
\pagestyle{fancy} % options: empty , plain , fancy
\renewcommand{\headrulewidth}{0pt} % customise the layout...
\lhead{}\chead{}\rhead{}
\lfoot{}\cfoot{\thepage}\rfoot{}

%%% SECTION TITLE APPEARANCE
\usepackage{sectsty}
\allsectionsfont{\sffamily\mdseries\upshape} % (See the fntguide.pdf for font help)
% (This matches ConTeXt defaults)

%%% ToC (table of contents) APPEARANCE
\usepackage[nottoc,notlof,notlot]{tocbibind} % Put the bibliography in the ToC
\usepackage[titles,subfigure]{tocloft} % Alter the style of the Table of Contents
\renewcommand{\cftsecfont}{\rmfamily\mdseries\upshape}
\renewcommand{\cftsecpagefont}{\rmfamily\mdseries\upshape} % No bold!

%%% END Article customizations

%%% The "real" document content comes below...


\title{Metody numeryczne zadanie nr 3}
\author{Mateusz Miotk \\  Sylwia Kaczmarczyk \\ Michał Kulesz}
%\date{} % Activate to display a given date or no date (if empty),
         % otherwise the current date is printed 

\begin{document}
\maketitle

\section{Treść zadania}
$\textbf{Zadanie 3.1} $ Zagadnienie różniczkowe: $y'=2y^2 -2x(x^3 - 1) , y(1)=1 $\\
rozwiązać na przedziale $[1,3]$ metodą Eulera oraz zmodyfikowaną metodą Eulera zwaną metodą punktu środkowego.\\
Wyniki porównać z rozwiązaniem dokładnym $y(x) = x^2$.

\section{Podstawy teoretyczne}
\subsection{Metoda Eulera}
Niech będzie dane równanie różniczkowe zwyczajne $y' = f(x,y(x))$ z warunkiem początkowym $y(x_0) = y_0$\\
Metoda Eulera polega na zastąpieniu krzywej całkowej $y = y(x)$ przechodzącej przez punkt $M_0(x_0,y_0)$, odpowiadający 
warunkom początkowym, łamaną $M_0,M_1,M_2,..,$ o wierzchołkach $M_i(x_i,y_i) , i=0,1,2,...,$ składającą się z odcinków prostych.\\
Wykorzystywane jest tutaj dane rownanie rekurencyjne: \\
%$\left\{\begin{array}  y_1 = y(x_0)+hf(x_0,y(x_0))\\y_{i+1} = y_i + hf(x_i,y_i)\end{array}}$
$$
 \left\{ \begin{array}{ll}
 y_0 = y(x_0)\\
 y_1 = y_0+hf(x_0,y(x_0))\\
y_{i+1} = y_i + hf(x_i,y_i)\\
\end{array} \right.
$$
\\gdzie h jest krokiem na osi x.\\
\subsection{Zmodyfikowana metoda Eulera}
Idea jest podobna ale wykorzystywany jest inny wzór rekurencyjny: \\
$$
 \left\{ \begin{array}{ll}
 y_0 = y(x_0)\\
 y_1 = y_0+hf(x_0 + \frac{h}{2},y_0 + f(x_0,y_0)\cdot\frac{h}{2})\\
 y_{i+1} = y(x_i)+hf(x_i + \frac{h}{2},y_i + f(x_i,y_i)\cdot\frac{h}{2})\\
\end{array} \right.
$$
\section{Algorytm realizujący zadanie}
\subsection{Algorytm}
\subsection{Przykładowe rozwiązanie}
\section{Opis programu}
\subsection{Opis struktur danych oraz funkcji w programie}
\subsection{Opis wejścia-wyjścia}
\subsection{Treść programu}
\subsection{Zrzuty wybranego programu}
\end{document}
