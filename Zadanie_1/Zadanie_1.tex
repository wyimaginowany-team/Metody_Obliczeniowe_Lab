% !TEX TS-program = pdflatex
% !TEX encoding = UTF-8 Unicode

% This is a simple template for a LaTeX document using the "article" class.
% See "book", "report", "letter" for other types of document.

\documentclass[16pt]{article} % use larger type; default would be 10pt

\usepackage[utf8]{inputenc} % set input encoding (not needed with XeLaTeX)
\usepackage[T1]{fontenc}
%%% Examples of Article customizations
% These packages are optional, depending whether you want the features they provide.
% See the LaTeX Companion or other references for full information.

%%% PAGE DIMENSIONS
\usepackage{geometry} % to change the page dimensions
\geometry{a4paper} % or letterpaper (US) or a5paper or....
% \geometry{margin=2in} % for example, change the margins to 2 inches all round
% \geometry{landscape} % set up the page for landscape
%   read geometry.pdf for detailed page layout information

\usepackage{graphicx} % support the \includegraphics command and options

% \usepackage[parfill]{parskip} % Activate to begin paragraphs with an empty line rather than an indent

%%% PACKAGES
\usepackage{booktabs} % for much better looking tables
\usepackage{array} % for better arrays (eg matrices) in maths
\usepackage{paralist} % very flexible & customisable lists (eg. enumerate/itemize, etc.)
\usepackage{verbatim} % adds environment for commenting out blocks of text & for better verbatim
\usepackage{subfig} % make it possible to include more than one captioned figure/table in a single float
% These packages are all incorporated in the memoir class to one degree or another...

%%% HEADERS & FOOTERS
\usepackage{fancyhdr} % This should be set AFTER setting up the page geometry
\pagestyle{fancy} % options: empty , plain , fancy
\renewcommand{\headrulewidth}{0pt} % customise the layout...
\lhead{}\chead{}\rhead{}
\lfoot{}\cfoot{\thepage}\rfoot{}

%%% SECTION TITLE APPEARANCE
\usepackage{sectsty}
\allsectionsfont{\sffamily\mdseries\upshape} % (See the fntguide.pdf for font help)
% (This matches ConTeXt defaults)

%%% ToC (table of contents) APPEARANCE
\usepackage[nottoc,notlof,notlot]{tocbibind} % Put the bibliography in the ToC
\usepackage[titles,subfigure]{tocloft} % Alter the style of the Table of Contents
\renewcommand{\cftsecfont}{\rmfamily\mdseries\upshape}
\renewcommand{\cftsecpagefont}{\rmfamily\mdseries\upshape} % No bold!

%%% END Article customizations

%%% The "real" document content comes below...

\title{Metody Obliczeniowe}
\author{Mateusz Miotk\\ Michał Kulesz\\ Sylwia Kaczmarczyk}
\date{} % Activate to display a given date or no date (if empty),
         % otherwise the current date is printed 

\begin{document}
\maketitle

\section{Treść zadania}

\textbf {Zadanie 1.14}: Ustalić naturalną $n_{max}$. Wczytać $ n \in \lbrace1,2,...,n_{max}\rbrace$, różne węzły $x_1.x_2,...,x_n $ oraz 
dowolne wartości $A_1,A_2,...,A_n$ i $B_1,B_2,...,B_n $. Wyznaczyć w postaci Newtona wielomian interpolacyjny Hermite'a $W = W(x)$ stopnia co najwyżej
$(2n-1)$ spełniający warunki: $W(x_i) = A_i$ oraz $W'(x_i) = B_i$ dla $i = 1,2,...,n$. Wynik przedstawić również w postaci ogólnej.

%\subsection{A subsection}
\section{Podstawa teoretyczna}
\subsection {Wielomian w postaci Newtona:}
Wielomian $p_k(x)$ można przedstawić w postaci:\\
$p_k(x) = \sum_{i=0}^{k} c_i \prod_{j=0}^{i-1} (x-x_j)$
Współczynniki $c_i$ to ilorazy różnicowe.

\subsection {Definicja ilorazów różnicowych}

Ogólnie liczbę $c_i$ definiujemy w następujący sposób:\\
$c_i = f[x_0,...,x_i] = \frac {f[x_1,x_2,...,x_{i-1}]-f[x_0,x_1,...,x_{i-1}]} {x_i - x_0}$
Jednak w naszych rozważaniach będziemy używać wzoru rekurencyjnego:\\
$c_{ij} = \frac {c_{i+1,j-1}- c_{i,j-1}} {x_{i+j}-x_i}$\\
Jeśli jednak wartość $c_i$ będzie wynosić $\frac{0}{0}$ to wpisujemy zamiast tego wartość pochodnej z $x_i$.
\\\\



\section{Algorytm, który ma realizować zadanie}
\subsection{Pobieranie danych.}
Na początku program zapyta nas o ilość RÓŻNYCH węzłów jakie chcemy wprowadzić do programu.Zostaną one wprowadzone do tablicy x[].\\
Następnie program zażąda od nas podania wartości funkcji w tych punktach. Zostaną one dodane do tablicy A[]. W tablicy A[] każda wartość zostanie podwójnie zapisanie w celu łatwiejszego 
policzenia tablicy różnic dzielonych.\\Następnie program zażąda podania wartości pochodnych w danych węzłach. Zapisane one będą do tablicy B[].\\
\subsection{Liczenie ilorazów różnicowych.}
W tym kroku wykorzystamy tablicę D[], która będzie miała tyle samo wyrazów co tablica A[].Wykorzystujemy to algorytm ,który wykorzystuje wzór rekurencyjny podany powyżej
a w wyniku otrzymamy wyłącznie jeden wiersz tablicę,który odpowiada wartością $c-_i$ wielomianu w postaci Newtona.\\
Jeżeli w wyniku obliczeń program napotka na działanie $\frac{0}{0}$ to w tym miejscu zapisywana jest wartość pochodnej w tym punkcie.

\subsection{Wypisanie wielomianu w postaci Newtona.}
Wykorzystywany jest wzór powyżej(Patrz 2.1).
\subsection{Wypisanie wielomianu w postaci ogólnej.}

\section{Przykładowe rozwiązanie dla małych danych}
Dla danych:\\
$p(1)=2$\\ $p'(1)=3$\\ $p(2)=6$\\ $p'(2)=7$\\
Tworzymy tabelę ilorazów różnicowych:\\
1 2|\textbf{\underline{3}} 1 2\\
1 2|4 3\\
2 6|\textbf{\underline{7}}\\
2 6|\\
Pogrubione i podkreślone wyrazy tablicy to wpisane wartości pochodnej w punkcie.\\
Wynika to, ponieważ:\\
$f[x_0,x_0]=\frac{f[x_0]-f[x_0]}{x_0-x_0} = \frac{2-2}{1-1} = \frac {0}{0} = f'(1) = 3$\\
$f[x_1,x_1]=\frac{f[x_1]-f[x_1]}{x_1-x_1} = \frac{6-6}{2-2} = \frac {0}{0} = f'(2) = 7$\\

Jest to pełna tabela. Program zapisze wyłącznie wiersz: \\
2 3 1 2 \\
co jest równoznaczne ze wspólczynnikami wielomianu w postaci Newtona.\\
Co daje wielomian postaci:\\
$p(x) = 2 + 3(x-1) + (x-1)^2 + 2(x-1)^{2}(x-2) $ - postać Newtona  \\
lub\\ $p(x) = 2x^3 - 7x^2 +11x -4$ w postaci ogólnej\\
$p'(x) = 6x^2 -14x +11$ \\
Sprawdzamy wyniki: \\
$p(1) = 2 + 3(1-1) + (1-1)^2 + 2(1-1)^{2}(1-2) = 2$\\
$p'(1) = 6-14+11=3$\\
$p(2) = 2 + 3(2-1) + (2-1)^2 + 2(2-1)^{2}(2-2) = 6$\\
$p'(x) = 6*4-14*2+11=24-28+11=-4+11=7$\\
\end{document}
